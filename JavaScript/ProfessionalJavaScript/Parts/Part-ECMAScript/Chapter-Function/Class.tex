\section{对象,类}

ECMA-262将对象定义为一组属性的无序集合。严格来说,这意味着对象就是一组没有特定顺序的值。对象的每个属性或方法都由一个名称来标识,这个名称映射到一个值。正因为如此(以及其他还未讨论的原因),可以把ECMAScript的对象想象成一张散列表,其中的内容就是一组名/值对,值可以是数据或者函数。

\subsection{对象}

创建自定义对象的通常方式是创建\texttt{Object}的一个新实例,然后再给它添加属性和方法:

\begin{JavaScript}
// 写法一
let person = new Object(); 
person.name = "Nicholas"; 
person.age = 29; 
person.job = "Software Engineer"; 
person.sayName = function() {   
    console.log(this.name);  
}; 
// 写法二
let person = {    
    name: "Nicholas",   
    age: 29,   
    job: "Software Engineer",   
    sayName() {     
        console.log(this.name);   
    } 
};
\end{JavaScript}

\subsubsection{属性的类型}

ECMA-262使用一些内部特性来描述属性的特征。为了将某个特性标识为内部特性,规范会用两个中括号把特性的名称括起来,比如\texttt{[[Enumerable]]}。

属性分为两种: 数据属性和访问器属性。

\noindent\textbf{数据属性}

数据属性包含一个保存数据值的位置。值会从这个位置读取,也会写入到这个位置。数据属性有4个特性描述它们的行为。

\begin{itemize}
    \item \texttt{[[Configuration]]}: 表示属性是否可以通过 \texttt{delete} 删除并重新定义,是否可以修改它的特性,以及是否可以把它改为访问器属性。默认为 \texttt{true}。
    \item \texttt{[[Enumerate]]}: 表示属性是否可以通过\texttt{for-in}循环返回。默认是\texttt{true}。
    \item \texttt{[[Writable]]}: 表示属性的值是否可以被修改。默认是\texttt{true}。
    \item \texttt{[[Value]]}: 包含属性实际的值。默认为\texttt{undefined}。
\end{itemize}

要修改属性的默认特性,就必须使用\texttt{Object.defineProperty}()方法。这个方法接收3个参数:要给其添加属性的对象、属性的名称和一个描述符对象。最后一个参数,即描述符对象上的属性可以包含:\texttt{configurable、enumerable、writable}和\texttt{value},跟相关特性的名称一一对应。根据要修改的特性,可以设置其中一个或多个值。比如:

\begin{JavaScript}
let person = {};  
Object.defineProperty(person, "name", {   
    writable: false,   
    value: "Nicholas" 
}); 
console.log(person.name); // "Nicholas" 
person.name = "Greg"; 
console.log(person.name); // "Nicholas" 
\end{JavaScript}

如果 \texttt{[[Configuration]]} 被设置为 \texttt{false},则不可以再通过 \texttt{defineProperty} 配置属性。

\noindent\textbf{访问器属性}

访问器属性不包含数据值。相反,它们包含一个获取(\texttt{getter})函数和一个设置(\texttt{setter})函数,不过这两个函数不是必需的。在读取访问器属性时,会调用获取函数,这个函数的责任就是返回一个有效的值。在写入访问器属性时,会调用设置函数并传入新值,这个函数必须决定对数据做出什么修改。访问器属性有4个特性描述它们的行为。

\begin{itemize}
    \item \texttt{[[Configuration]]}: 表示属性是否可以通过 \texttt{delete} 删除并重新定义,是否可以修改它的特性,以及是否可以把它改为访问器属性。默认为 \texttt{true}。
    \item \texttt{[[Enumerate]]}: 表示属性是否可以通过\texttt{for-in}循环返回。默认是\texttt{true}。
    \item \texttt{[[Get]]}: 获取函数,在读取属性时调用。默认值为\texttt{undefined}。
    \item \texttt{[[Set]]}: 设置函数,在写入属性时调用。默认值为\texttt{undefined}。
\end{itemize}

访问器属性是不能直接定义的,必须使用\texttt{Object.defineProperty()}。

\begin{JavaScript}
let book = {    
    year_: 2017,   
    edition: 1
}; 

Object.defineProperty(book, "year", {
    get() {     
        return this.year_;   
    },   
    set(newValue) {     
        if (newValue > 2017) {       
            this.year_ = newValue;       
            this.edition += newValue - 2017;     
        }   
    } 
}); 
book.year = 2018; 
console.log(book.edition); // 2 
\end{JavaScript}

同时,ECMAScript提供了 \texttt{Object.defineProperties()} 方法来同时定义多个属性:

\begin{JavaScript}
let book = {};  
Object.defineProperties(book, {   
    year_: {     
        value: 2017   
    },   
    edition: {     
        value: 1   
    },   
    year: {     
        get() {       
            return this.year_;     
        }, 
        set(newValue) {       
            if (newValue > 2017) {         
                this.year_ = newValue;         
                this.edition += newValue - 2017;       
            }     
        }   
    } 
}); 
\end{JavaScript}

使用\texttt{Object.getOwnPropertyDescriptor()}方法可以取得指定属性的属性描述符。这个方法接收两个参数:属性所在的对象和要取得其描述符的属性名。返回值是一个对象,对于访问器属性包含\texttt{configurable、enumerable、get}和\texttt{set}属性,对于数据属性包含\texttt{configurable、enumerable、writable}和\texttt{value}属性。

ES6 也提供了访问多个属性的方法: \texttt{Object.getOwnPropertyDescriptors()}。

\subsubsection{对象操作}

\noindent\textbf{\texttt{Object.assign()}}

ES6 专门为合并对象提供了\texttt{Object.assign()}方法。这个方法接收一个目标对象和一个或多个源对象作为参数,然后将每个源对象中可枚举(\texttt{Object.propertyIsEnumerable()}返回\texttt{true})和自有(\texttt{Object.hasOwnProperty()}返回\texttt{true})属性复制到目标对象。以字符串和符号为键的属性会被复制。对每个符合条件的属性,这个方法会使用源对象上的\texttt{[[Get]]}取得属性的值,然后使用目标对象上的\texttt{[[Set]]}设置属性的值。

如果赋值期间出错,则操作会中止并退出,同时抛出错误。\texttt{Object.assign()没}有“回滚”之前赋值的概念,因此它是一个尽力而为、可能只会完成部分复制的方法。

\noindent\textbf{对象解构}

ECMAScript 6新增了对象解构语法,可以在一条语句中使用嵌套数据实现一个或多个赋值操作。简单地说,对象解构就是使用与对象匹配的结构来实现对象属性赋值。

\begin{JavaScript}
let person = {   
    name: 'Matt',   
    age: 27 
}; 
let { name: personName, age: personAge } = person; 
console.log(personName);  // Matt 
console.log(personAge);   // 27 
\end{JavaScript}

使用解构,可以在一个类似对象字面量的结构中,声明多个变量,同时执行多个赋值操作。如果想让变量直接使用属性的名称,那么可以使用简写语法,比如:

\begin{JavaScript}
let person = {    
    name: 'Matt',   
    age: 27 
}; 
let { name, age, job } = person; 
console.log(name);  // Matt 
console.log(age);   // 27 
console.log(job);   // undefined 
\end{JavaScript}

解构在内部使用函数\texttt{ToObject()}(不能在运行时环境中直接访问)把源数据结构转换为对象。这意味着在对象解构的上下文中,原始值会被当成对象。这也意味着(根据\texttt{ToObject()}的定义),\texttt{null}和\texttt{undefined}不能被解构,否则会抛出错误。

解构对于引用嵌套的属性或赋值目标没有限制。为此,可以通过解构来复制对象属性:

\begin{JavaScript}
let person = {    
    name: 'Matt',   
    age: 27,   
    job: {     
        title: 'Software engineer'   
    } 
}; 
let personCopy = {}; 

({   
    name: personCopy.name,   
    age: personCopy.age,   
    job: personCopy.job 
} = person); 

person.job.title = 'Hacker' 
console.log(person);        // { name: 'Matt', age: 27, job: { title: 'Hacker' } } 
console.log(personCopy);    // { name: 'Matt', age: 27, job: { title: 'Hacker' } } 
\end{JavaScript}

\subsection{创建对象}

\subsubsection{工厂模式鱼构造函数模式}

虽然使用\texttt{Object}构造函数或对象字面量可以方便地创建对象,但这些方式也有明显不足:创建具有同样接口的多个对象需要重复编写很多代码。

\noindent\textbf{工厂模式}

\begin{JavaScript}
function createPerson(name, age, job) {   
    let o = new Object();    
    o.name = name;   
    o.age = age;   
    o.job = job;   
    o.sayName = function() {     
        console.log(this.name);   
    };   
    return o; 
} 
let person1 = createPerson("Nicholas", 29, "Software Engineer"); 
let person2 = createPerson("Greg", 27, "Doctor");
\end{JavaScript}

这里,函数\texttt{createPerson()}接收3个参数,根据这几个参数构建了一个包含\texttt{Person}信息的对象。可以用不同的参数多次调用这个函数,每次都会返回包含3个属性和1个方法的对象。这种工厂模式虽然可以解决创建多个类似对象的问题,但没有解决对象标识问题(即新创建的对象是什么类型)。

\noindent\textbf{构造函数模式}

前面的例子使用构造函数模式可以这样写:

\begin{JavaScript}
function Person(name, age, job){    
    this.name = name;   
    this.age = age;   
    this.job = job;   
    this.sayName = function() {     
        console.log(this.name);   
    }; 
} 
let person1 = new Person("Nicholas", 29, "Software Engineer"); 
let person2 = new Person("Greg", 27, "Doctor"); 
person1.sayName();  // Nicholas 
person2.sayName();  // Greg 
\end{JavaScript}

在这个例子中,\texttt{Person()}构造函数代替了\texttt{createPerson()}工厂函数。实际上,\texttt{Person()}内部的代码跟\texttt{createPerson()}基本是一样的,只是有如下区别。

\begin{itemize}
    \item 没有显式地创建对象。
    \item 属性和方法直接赋值给了\texttt{this}。
    \item 没有\texttt{return}。
\end{itemize}

另外,要注意函数名\texttt{Person}的首字母大写了。按照惯例,构造函数名称的首字母都是要大写的,非构造函数则以小写字母开头。这是从面向对象编程语言那里借鉴的,有助于在ECMAScript中区分构造函数和普通函数。毕竟ECMAScript的构造函数就是能创建对象的函数。

要创建\texttt{Person}的实例,应使用\texttt{new}操作符。以这种方式调用构造函数会执行如下操作。

\begin{itemize}
    \item 在内存中创建一个新对象。
    \item 这个新对象内部的\texttt{[[Prototype]]}特性被赋值为构造函数的\texttt{prototype}属性。
    \item 构造函数内部的\texttt{this}被赋值为这个新对象(即\texttt{this}指向新对象)。
    \item 执行构造函数内部的代码(给新对象添加属性)。
    \item 如果构造函数返回非空对象,则返回该对象;否则,返回刚创建的新对象。
\end{itemize}

上一个例子的最后,\texttt{person1}和\texttt{person2}分别保存着\texttt{Person}的不同实例。这两个对象都有一个\texttt{constructor}属性指向\texttt{Person},如下所示:

\begin{JavaScript}
console.log(person1.constructor == Person);  // true 
console.log(person2.constructor == Person);  // true 
\end{JavaScript}

在实例化时,如果不想传参数,那么构造函数后面的括号可加可不加。只要有new操作符,就可以调用相应的构造函数。

构造函数与普通函数唯一的区别就是调用方式不同。除此之外,构造函数也是函数。并没有把某个函数定义为构造函数的特殊语法。任何函数只要使用new操作符调用就是构造函数,而不使用new操作符调用的函数就是普通函数。

\subsubsection{原型模式}

每个函数都会创建一个\texttt{prototype}属性,这个属性是一个对象,包含应该由特定引用类型的实例共享的属性和方法。实际上,这个对象就是通过调用构造函数创建的对象的原型。使用原型对象的好处是,在它上面定义的属性和方法可以被对象实例共享。原来在构造函数中直接赋给对象实例的值,可以直接赋值给它们的原型,如下所示:

\begin{JavaScript}
function Person() {} 
Person.prototype.name = "Nicholas"; 
Person.prototype.age = 29; 
Person.prototype.job = "Software Engineer"; 
Person.prototype.sayName = function() {   
    console.log(this.name);  
}; 
let person1 = new Person(); 
person1.sayName(); // "Nicholas" 
let person2 = new Person(); 
person2.sayName(); // "Nicholas" 
console.log(person1.sayName == person2.sayName); // true 
\end{JavaScript}


\noindent\textbf{理解原型}

无论何时,只要创建一个函数,就会按照特定的规则为这个函数创建一个\texttt{prototype}属性(指向原型对象)。默认情况下,所有原型对象自动获得一个名为\texttt{constructor}的属性,指回与之关联的构造函数。因构造函数而异,可能会给原型对象添加其他属性和方法。

每次调用构造函数创建一个新实例,这个实例的内部\texttt{[[Prototype]]}指针就会被赋值为构造函数的原型对象。实例与构造函数原型之间有直接的联系,但实例与构造函数之间没有。

在 JavaScript 内部,和实例对象相关的有三个对象:
\begin{itemize}
    \item 构造函数: 我们定义的,一般为首字母大写的的函数。包含如下成员:
    \begin{itemize}
        \item \texttt{prototype}: 指向函数原型。
    \end{itemize}
    \item 函数原型: 我们定义构造函数后,自动生成的,用于存储数据。包含如下成员:
    \begin{itemize}
        \item \texttt{constructor}: 指向构造函数。
        \item 构造函数定义的成员。
    \end{itemize}
    \item 实例对象: 通过构造函数创建的实例对象,保存一个 \texttt{[[prototype]]} 属性指向函数原型。可以通过 \texttt{\_\_proto\_\_} 属性访问。
\end{itemize}

也就是说,我们在创建一个函数时,JavaScript 内部会创建这个函数对应的原型,函数本身只保存指向这个原型的指针,通过函数创建对象实例时本质上时通过原型创建实例。

\noindent\textbf{原型的问题}

原型模式弱化了向构造函数传递初始化参数的能力,会导致所有实例默认都取得相同的属性值。虽然这会带来不便,但还不是原型的最大问题。原型的最主要问题源自它的共享特性。

这对于函数来说没有什么问题,函数一般不会也不建议保存数据。而对于属性来说,则要注意不要轻易修改原型中的属性,这会导致所有使用该原型的实例对象保存的属性均被改变。

\subsection{继承}

\subsubsection{原型链继承}

我们知道,ECMAScript 是没有类的,所有对象均源自原生的 \texttt{Object} 类型。本质上,ECMAScript 的核心设计思想:原型 代替了 传统语言的类与继承思想。这两者在具体实现上不同,但实现的效果类似。简而言之,ECMAScript 通过原型链实现类的继承,并且用法类似。

默认情况下,所有引用对象都继承自 \texttt{Object},因此自定义构造函数的原型也指向 \texttt{Object} 的 \texttt{Prototype}。 

确认继承关系有两种方式,一种是通过原生的 \texttt{instanceof} 操作符,另一种时通过原型的 \texttt{isPrototypeOf} 方法。

由于没有类,要通过原型实现继承,因此需要手动指定 \texttt{prototype} 来实现继承:

\begin{JavaScript}
function SuperType() {    
    this.property = true; 
} 
SuperType.prototype.getSuperValue = function() {   
    return this.property; 
}; 
function SubType() {   
    this.subproperty = false; 
} 

// 继承SuperType 
SubType.prototype = new SuperType(); 

// 新方法
SubType.prototype.getSubValue = function () { 
    return this.subproperty; 
}; 

// 覆盖已有的方法
SubType.prototype.getSuperValue = function () { 
    return false; 
}; 
let instance = new SubType(); 
console.log(instance.getSuperValue()); // false 
\end{JavaScript}

\subsubsection{经典继承}

这样继承存在一个问题,原型包含引用值,对原型引用值得修改将影响所有与之关联的对象。为了解决这个问题,引入了 ``盗用构造函数''(也称``经典继承'')。

\begin{JavaScript}
function SuperType() {    
    this.colors = ["red", "blue", "green"]; 
} 

function SubType() { 
    // 继承SuperType 
    SuperType.call(this); 
} 

let instance1 = new SubType(); 
instance1.colors.push("black"); 
console.log(instance1.colors); // "red,blue,green,black" 
let instance2 = new SubType(); 
console.log(instance2.colors); // "red,blue,green" 
\end{JavaScript}

通过使用\texttt{call()}(或\texttt{apply()})方法,\texttt{SuperType}构造函数在为\texttt{SubType}的实例创建的新对象的上下文中执行了。这相当于新的\texttt{SubType}对象上运行了\texttt{SuperType()}函数中的所有初始化代码(不包括函数)。结果就是每个实例都会有自己的\texttt{colors}属性。

如果需要传参,在 \texttt{call()} 的 \texttt{this} 变量后传入其他参数即可。

盗用构造函数的主要缺点,也是使用构造函数模式自定义类型的问题:必须在构造函数中定义方法,因此函数不能重用。此外,子类也不能访问父类原型上定义的方法,因此所有类型只能使用构造函数模式。由于存在这些问题,盗用构造函数基本上也不能单独使用。

\subsubsection{组合继承}

组合继承(有时候也叫伪经典继承)综合了原型链和盗用构造函数,将两者的优点集中了起来。基本的思路是使用原型链继承原型上的属性和方法,而通过盗用构造函数继承实例属性。这样既可以把方法定义在原型上以实现重用,又可以让每个实例都有自己的属性。来看下面的例子:

\begin{JavaScript}
function SuperType (name){
    this.name = name;
    this.colors ["red","blue","green"];
}
    
SuperType.prototype.sayName = function(){
    console.log(this.name);
};

function SubType(name,age){
    //继承属性
    SuperType.call(this,name);
    this.age = age;
}

//继承方法
SubType.prototype = new SuperType();
SubType.prototype.sayAge = function () {
    console.log(this.age);
}
\end{JavaScript}

调用 \texttt{call()} 方法本质上时将成员属性都执行一遍,因此 \texttt{SubType} 有了父类的所有属性,而 \texttt{SubType.prototype = new SuperType();} 则表示,原型链上一级为 \texttt{SuperType} 的原型。由于我们已经相当于写过一边成员属性,所以调用时会优先使用 \texttt{SubType} 的成员属性。

这是目前使用最广泛的方法。

\subsection{类}

从前一节可以看出,ECMAScript 通过原型实现对象继承功能写出的代码比较冗余。在 Java,Python 中简单且常用的继承在 JS 中一点都不方便。为此,ES6 提供了正宗的 \texttt{class} 关键字用于定义类,虽然本质上还是原型链。

ECMAScript 中的类与函数类似,但有以下不同:
\begin{itemize}
    \item 无法提升。
    \item 作用域是块作用域,而不是函数作用域。
\end{itemize}

同样可以用两种方式创建类:
\begin{JavaScript}
class Person {};
const Animal = class {}; 
\end{JavaScript}

\noindent\textbf{构造函数}

类中的 \texttt{constructor} 函数是类的构造函数,解释器在调用 \texttt{new} 操作时会调用这个函数。构造函数不是必要的,不定义的话构造函数就是空函数。

使用\texttt{new}调用类的构造函数会执行如下操作:
\begin{itemize}
    \item 在内存中创建一个新对象。
    \item 这个新对象内部的\texttt{[[Prototype]]}指针被赋值为构造函数的\texttt{prototype}属性。
    \item 构造函数内部的\texttt{this}被赋值为这个新对象(即\texttt{this}指向新对象)。
    \item 执行构造函数内部的代码(给新对象添加属性)。
    \item 如果构造函数返回非空对象,则返回该对象;否则,返回刚创建的新对象。
\end{itemize}

类实例化时传入的参数会用作构造函数的参数。如果不需要参数,则类名后面的括号也是可选的:

\begin{JavaScript}
class Person {
    constructor(name) {
        console.log(arguments.length);
        this.name = name || null;
    }
}
let p1 = new Person;            // 0 
console.log(p1.name);           // null 
let p2 = new Person();          // 0 
console.log(p2.name);           // null 
let p3 = new Person('Jake');    // 1 
console.log(p3.name);           // Jake 
\end{JavaScript}

类构造函数与构造函数的主要区别是,调用类构造函数必须使用\texttt{new}操作符。而普通构造函数如果不使用\texttt{new}调用,那么就会以全局的\texttt{this}(通常是\texttt{window})作为内部对象。调用类构造函数时如果忘了使用\texttt{new}则会抛出错误:

\noindent\textbf{类不同于函数}

类本质上还是函数,可以使用函数相关的所有操作,但是类内部的成员则有所不同。

每次通过new调用类标识符时,都会执行类构造函数。在这个函数内部,可以为新创建的实例(\texttt{this})添加“自有”属性。至于添加什么样的属性,则没有限制。另外,在构造函数执行完毕后,仍然可以给实例继续添加新成员。每个实例都对应一个唯一的成员对象,这意味着所有成员都不会在原型上共享。

但是,也可以通过类定义语句把类块中定义的方法作为原型方法,实现实例间共享:

\begin{JavaScript}
class Person {
    constructor() {     // 添加到this的所有内容都会存在于不同的实例上    
        this.locate = () => console.log('instance');
    } 
    // 在类块中定义的所有内容都会定义在类的原型上
    locate() { console.log('prototype'); }
} let p = new Person();
p.locate();                 // instance 
Person.prototype.locate();  // prototype 
\end{JavaScript}

类中可以有静态方法,但每个类只能有一个静态成员:

\begin{JavaScript}
class Person {
    constructor() {     // 添加到this的所有内容都会存在于不同的实例上    
        this.locate = () => console.log('instance');
    } 
    // 在类块中定义的所有内容都会定义在类的原型上
    locate() {
        console.log('prototype');
    }
    // 定义在类本身上  
    static locate() {
        console.log('class', this);
    } 
}
\end{JavaScript}

此外,也可以像函数一样在类定义外增加函数:

\begin{JavaScript}
// 在类上定义独享的函数
Person.greeting = 'My name is'; 
// 在原型上定义共享的函数
Person.prototype.name = 'Jake';
\end{JavaScript}

\noindent\textbf{继承}

有类自然有继承,ES6 同时提供了 \texttt{extends} 关键字用于类继承:
\begin{itemize}
    \item ES6 规定 JS 是单继承。
    \item 提供了 \texttt{super} 方法:
    \begin{itemize}
        \item 仅限于派生类构造方法和静态方法内部调用。
        \item 在构造函数中使用 \texttt{super} 用于调用父类构造函数,将实例返回给 \texttt{this};且 \texttt{this} 必须在 \texttt{super} 之后调用。如果没有定义构造函数,在实例化派生类时会自动调用。
    \end{itemize}
\end{itemize}

\noindent\textbf{抽象基类}

有时候可能需要定义这样一个类,它可供其他类继承,但本身不会被实例化。虽然ECMAScript没有专门支持这种类的语法,但通过\texttt{new.target}也很容易实现。\texttt{new.target}保存通过\texttt{new}关键字调用的类或函数。

\begin{JavaScript}
class Vehicle {
    constructor() {
        console.log(new.target);
        if (new.target === Vehicle) {
            throw new Error('Vehicle cannot be directly instantiated');
        }
    }
}
\end{JavaScript}

\newpage