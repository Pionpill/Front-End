\section{函数}

ECMAScript 中,每个函数都是 \texttt{Function} 类型的实例\footnote{与 Python 一样。}。而 \texttt{Function} 也有属性和方法,跟其他引用类型一样。函数名就是指向函数对象的指针,而且不一定与函数本身密切绑定。函数通常以函数声明的方式定义:

\begin{JavaScript}
function sum(num1, num2) {
    return num1 + num2;
}
\end{JavaScript}

另一种定义函数的语法是函数表达式。函数表达式与函数声明几乎是等价的:

\begin{JavaScript}
let sum = function(num1, num2) {
    return num1 + num2;
};
\end{JavaScript}

注意函数表达式最后是要带分号的,而函数声明则没有必要。

还有一种定义函数的方式与函数表达式很想,叫``箭头函数'':

\begin{JavaScript}
let sum = (num1, num2) => {
    return num1 + num2;
};
\end{JavaScript}

最后一种定义函数的方式是使用 \texttt{Function} 构造函数。这个构造函数手机任意多个字符串参数,最后一个参数为函数体,之前的参数都是新函数的参数:

\begin{JavaScript}
let sum = new Function("num1", "num2", "return num1 + num2");
\end{JavaScript}

这种方式并不推荐,会影响内部处理的性能。不过要记作 ECMAScript 中的函数是对象,函数名是指针。

\subsection{箭头函数}

箭头函数是 ES6 新增的语法。箭头函数和函数表达式十分相像,任何可以使用函数表达式的地方,都可以使用箭头函数。

箭头函数的语法非常适合嵌入函数的场景:

\begin{JavaScript}
let ints = [1,2,3];
console.log(ints.map(function(i) {return i+1;}));   // [2,3,4]
console.log(ints.map((i) => {return i+1;}));   // [2,3,4]
\end{JavaScript}

如果只有一个参数,那可以不用括号。

\begin{JavaScript}
let double = (x) => {return 2*x;};
let triple = x => {return 3*x;};
\end{JavaScript}

如果函数语句十分精简,只需要一行代码,那么大括号也可以省略。省略大括号会隐式返回这些代码的值(不需要额外的 \texttt{return} 语句)。

\begin{JavaScript}
let double = x => 2 * x;
\end{JavaScript}

箭头函数虽然语法简洁,但也有很多场合不适合。箭头函数不能使用 \texttt{arguments , super} 和 \texttt{new.target},也不能用作构造函数。此外,箭头函数也没有 \texttt{prototype} 属性。

\subsection{函数名}

因为函数名是指向函数的指针,所以它们跟其他包含对象指针的变量具有相同行为。这意味着一个函数可以拥有多个名称。

ECMAScript6 的所有函数对象都会暴露一个和只读的 \texttt{name} 属性,其中包含关于函数的信息。多数情况下,这个属性保存的就是一个函数标识符,或者说是一个字符串化的变量名。如果它是使用 \texttt{Function} 构造函数创建的,则会标识为 ``anonymous''。

\begin{JavaScript}
function foo() {}
let bar = function() {};
let baz = () => {};

console.log(foo.name);      // foo
console.log(bar.name);      // bar
console.log(baz.name);      // baz
console.log(() => {}.name);      // 空字符串
console.log((new Function()).name);      // anonymous
\end{JavaScript}

如果一个函数是获取函数,设置函数,或者使用 \texttt{bind()} 实例化,那么标识符前面会加上一个前缀\footnote{后文会详细说明}。

\subsection{理解函数}

ECMAScript 的函数去其他大多数语言不同,它既不关心参数的类型也不关心参数的个数,甚至传入参数的个数可以超过定义函数时参数的个数。

之所以会这样,是因为 ECMAScript 函数的参数在内部表现为一个数数组。函数被调用时总会接收一个数组,但函数并不关系这个数组中有什么(甚至什么都没有)。

事实上,在使用 \texttt{function} 关键字定义(非箭头)函数时,可以在函数内部访问 \texttt{arguments} 对象,从中取得传进来的每个参数值。我们可以通过中括号语法访问 \texttt{arguments} 对象中的元素(第一个是 \texttt{arguments[0]})。而要确定传进来多少个参数,可以访问 \texttt{arguments.length} 属性。

\begin{JavaScript}
function sayHi(name, message) {
    console.log("Hello" + name + "," + message);
    }
    
// 等价写法
function sayHi() {
    console.log("Hello" + arguments[0] + "," + arguments[1]);
}
\end{JavaScript}

上面例子表明,ECMAScript 函数的参数只是为了方便才写出来的,并不是必须写出来。与此同时,ECMAScript 中的命名参数不会创建让之后的调用必须匹配的函数签名。这是因为根本不存在验证命名参数的机制。

\newpage