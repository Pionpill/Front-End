\documentclass{PionpillNote-book}

\title{JavaScript 高级程序设计 笔记}
\author{
    Pionpill \footnote{笔名:北岸,电子邮件:673486387@qq.com,Github:\url{https://github.com/Pionpill}} \\
    本文档为作者学习《JavaScript 高级程序设计》\footnote{《Professional JavaScript for Web Developers》:Matt Frisbie 2020年9月第2版}一书时的笔记。\\
}

\date{\today}

\begin{document}

\pagestyle{plain}
\maketitle

\noindent\textbf{前言:}

笔者为软件工程系在校本科生,主要用 JavaScript 做一些前端项目。

<<JavaScript 高级程序设计>> 是 JavaScript 学习的一本入门及进阶书,原书中文版多达117万字,笔者将其分为三个部分:
\begin{itemize}
    \item JavaScript 基础:数据结构,分支语句,函数,面向对象等基础知识
    \item JavaScript 进阶:异步,多线程,DOM,BOM,JSON/XML 处理等常用知识
    \item JavaScript 前沿:ES6 之后的新特性。
\end{itemize}

本笔记不能代替原书,仅是对原书的一个总结归纳,笔记上只有知识点的总结,并没有详细的理解性语句,如有需要,还请购买原书。本文所在Github仓库采用 GPL v3 协议,但请勿将本文商业使用,本文引用了一些 CSDN 或其他论讨的文章,如果原作者觉得不合适,请联系本人。

此外,关于 Web 开发,避不开又令所有技术员头疼的是各种历史遗留问题,以及不同浏览器兼容问。,由于笔者仅为一名大三学生,预计读完研究生后才会真正工程性地利用这些技术\footnote{希望到时候没这些问题}。所以对于这些历史与兼容问题,除非特别重要,笔记中不再详述。这些问题包括:
\begin{itemize}
    \item IE 浏览器适配
    
    微软已经改用 Chrome 核心的 Edge 浏览器了,本文直接屏蔽 IE 浏览器。
    \item W3C 规则
    
    早期 JS 发展并没有统一的标准,这段时间造成的历史问题。
    \item 淘汰技术
    
    部分已被淘汰(正在淘汰)的技术,比如 XHTML,CSS标签的部分不常用属性。
\end{itemize}

本人的编写及开发环境如下:
\begin{itemize}
    \item IDE: VSCode 1.62
    \item Chrome: 91.0
    \item Node.js: 14.16
    \item Window10: 10.0.19042
\end{itemize}

此外,JavaScript 是一门语法十分宽松的语言,其语法在某些方面甚至比 Python 都要自由,但考虑到可读性,笔记中只强调推荐的写法,至于一些可行但不推荐的写法,仅提示。

\date{\today}
\tableofcontents
\newpage

\setcounter{page}{1} 
\pagestyle{fancy}


\import{Parts/Part-1/Chapter-1}{section-1.tex}


\end{document}

