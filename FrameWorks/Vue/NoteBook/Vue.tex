\documentclass{PionpillNote-book}

\title{Vue 笔记}
\author{
    Pionpill \footnote{笔名:北岸,电子邮件:673486387@qq.com,Github:\url{https://github.com/Pionpill}} \\
    本文档为作者学习前端框架 React 时的笔记。\\
}

\date{\today}

\begin{document}

\pagestyle{plain}
\maketitle

\noindent\textbf{前言:}

笔者为软件工程系在校本科生,出于兴趣学习 Vue。主要学习途径\footnote{由于市面上没什么好的 Vue 书,直接看了官方文档}为 Vue 官网教程 \footnote{\url{https://v3.cn.vuejs.org/guide/installation.html}}。

笔记内容大都为笔者自学过程中总结的一些知识,由于没有书籍支撑,部分内容可能并不是十分权威。因此,我会在每一小节前面附上原文的链接(一般为中文文档,可能是英文版本)。

本人的编写及开发环境如下:
\begin{itemize}
    \item IDE: VSCode 1.62
    \item Chrome: 91.0
    \item Node.js: 14.16
    \item Window10: 10.0.19042
    \item Vue3: vue/cli 4.5.15
\end{itemize}

Vue 是一个相对容易上手(相对 React)的前端框架,适合中小型企业的轻量级网站搭建。不过,在阅读本文之前,你必须了解 Html,CSS,JavaScript 的中阶语法。

此外,本文不适合入门 Vue,只是作为笔者的笔记,忘记一些 Vue 语法时方便查阅,要入门学习真得看官方文档。

本书的一些主要参考与引用资料:
\begin{itemize}
    \item Vue3 官方教程: \url{https://v3.cn.vuejs.org}
    \item 黑马程序员入门教程\footnote{虽然是基于 Vue2,但入门强烈推荐看这个。}: \url{https://www.bilibili.com/video/BV12J411m7MG}
\end{itemize}

\date{\today}
\newpage

\tableofcontents

\newpage

\setcounter{page}{1} 
\pagestyle{fancy}


\import{Chapters/Chapter-1}{Section-1.tex}
\import{Chapters/Chapter-1}{Section-2.tex}
\import{Chapters/Chapter-1}{Section-3.tex}


\end{document}

