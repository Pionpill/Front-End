\chapter{Vue 基础与快速入门}

这章只学一些浅显的内容,方便入门。

\section{安装与创建项目}
在安装 Vue 之前,要确保已经安装过 Node.js 和 npm(npm 内嵌在 Node.js 中)。

Vue 是渐进式的 JavaScript 框架,这意味着它可以根据需求以多种方式集成到一个项目中。主要的四种方式为:

\begin{itemize}
    \item 在页面上以 CDN 包的形式导入。
    \item 下载 JavaScript 文件并自行托管。
    \item 使用 npm 安装它。
    \item 使用官方 CLI 来构建一个项目。
\end{itemize}

下面仅介绍主流的 npm 安装 vue3 方法\footnote{其它方法参考: \url{https://v3.cn.vuejs.org/guide/installation.html}}。

\subsection{npm 安装 vue3}

使用下面指令可以安装 vue3。

\begin{bash}
# 安装最新稳定版
npm install vue@next
# 安装完成后查看版本
vue -V
\end{bash}

Vue 还提供了编写单文件组件的配套工具。如果你想使用单文件组件,那么你还需要安装:

\begin{bash}
npm install -D @vue/compiler-sfc
\end{bash}

\subsection{Vue 的一些说明}

Vue 是渐进式框架,因此可以和原始的 HTML 文件并存。此外所有使用 Vue 语法编写的代码或文件都能转换为 Html 代码。

Vue 除了自己的语法,同时支持 JSX 等多种语法,往往语法不一样,但实现效果是相同的。

\subsection{CDN 引用}
Vue 可以当都引入一个 Html 文件,只需加入如下代码(二选一):
\begin{HTML}
<!-- 开发环境版本,包含了有帮助的命令行警告 -->
<script src="https://cdn.jsdelivr.net/npm/vue@2/dist/vue.js"></script>
<!-- 生产环境版本,优化了尺寸和速度 -->
<script src="https://cdn.jsdelivr.net/npm/vue@2"></script>
\end{HTML}

\subsection{创建 Vue 项目}
\subsubsection{图形化界面}

Vue3 提供了图形化创建项目的方式,通过下面指令进入:

\begin{bash}
vue ui
\end{bash}

进入项目管理,按需求创建项目(ui 界面跟着做就行,不再多介绍)。

\subsubsection{命令行创建}

\texttt{cd} 到需要创建项目的位置后,输入如下指令:
\begin{bash}
vue init webpack project
\end{bash}

其中 \texttt{project} 是项目名,可以自行命名。\texttt{webpack} 是以 \texttt{webpack} 为模板生成项目,还可以是其他模板。在下载好必要的包后,会提示一连串内容供开发者选择,大部分不是 Vue 所必需的,多为辅助构建项目,或者格式化代码等,这都不是 Vue 框架内的知识,请读者自行查阅资料了解。
\newpage