\section{模板基础语法}

这一章讲解一些初级语法的初级用法,无需构建 Vue 项目,直接在 Html 中即可书写。

\subsection{Vue 对象与绑定}

我们在脚本中创建的 Vue 对象可以通过 \texttt{el} 挂载点(Vue 对象的一个属性)的值绑定到元素上,\texttt{el} 的值是 CSS 的任意类型选择器。在 Vue 对象中的 \texttt{data} 属性的数据可以通过插值表达式传入 Html 页面中,插值表达式的形式为: \{\{data 字典中的元素\}\}。

\lstinputlisting[language=HTML]{../../Examples/Basic/el.html}

通过 id 选择器,我们将 Vue 对象 \texttt{app} 绑定到了页面中的一个 \texttt{div} 块中,通过插值表达式我们将数据显示在了 Html 页面上\footnote{请读者自行在 Examples 文件中找对应 Html 文件查看效果。}。这就是 Vue 的最基础的工作方式。

\subsubsection{Vue 对象}

Vue 对象的构造函数需传入一个字典,字典中的键是确定的,每个键都有自己的意义\footnote{将在后文详细说明}。

\subsubsection{\texttt{el} 属性(挂载点)}

在 \texttt{el} 命中的元素的内部,可以进行任意嵌套。也即所有 Vue 对象都在对应命中的 DOM 节点及其子结点中起作用。

el 挂载点的值是 css 中的任意类型选择器,例如 \texttt{"\#app", ".app"} 分别表示 \texttt{id,class} 值为 \texttt{app} 的 Html 标签。但在实际应用中一般为 \texttt{id} 选择器。

挂载点可以绑定任意类型的 Html 双标签(\texttt{html} 与 \texttt{body} 标签除外),不仅限于示例中的 \texttt{div} 标签。在实际开发过程中,多使用 \texttt{div} 标签。

\subsubsection{\texttt{data} 属性}

\texttt{data} 顾名思义是数据属性,其值可以是基础类型或数组,字典等复杂类型。基础数据类型直接通过插值表达式写入键名即可。

复杂数据类型如果直接写入键名会渲染对应的原始值(包含一些括号,引号等),因此我们需要用 JavaScript 的语法获取对应的基础数据类型值。比如在数组中使用中括号语法,在字典中使用点语法。

\lstinputlisting[language=HTML]{../../Examples/Basic/data.html}

\subsection{Vue 指令}
\subsubsection{内容绑定: \texttt{v-text, v-html, v-on}}
\noindent\textbf{\texttt{v-text}}

\texttt{v-text} 指令用于设置标签的文本值,该指令对应的值为 Vue 对象 \texttt{data} 属性中的键。
\begin{itemize}
    \item \texttt{v-text} 指令的值将替代标签内其他内容,也即无效化标签内的内容。
    \item 如果要保留标签内的其他内容,请使用插值表达式。
\end{itemize}

\begin{HTML}
<div id = "app">
    <h2 v-text="message"> </h2>     <!--显示 message 值-->
    <h2> 阿巴阿巴 {{message}} </h2>     <!--显示 '阿巴阿巴'和 message 值-->
</div>
\end{HTML}

插值表达式和 \texttt{v-text} 指令均支持基础的表达式。

\begin{HTML}
<div id = "app">
    <h2 v-text="message+'!'"> </h2>     
    <h2> 阿巴阿巴 {{message+"!"}} </h2>     
</div>
\end{HTML}

\noindent\textbf{\texttt{v-html}}

基础语法和 \texttt{v-text} 相同,如果是普通文本,显示也相同。

如果内容中是 Html 结构,则会被解析出来。

\lstinputlisting[language=HTML]{../../Examples/Basic/v-html.html}

\noindent\textbf{\texttt{v-on}}

\texttt{v-on} 指令用于绑定事件。由于 \texttt{v-on} 已经有了 on,对于 Html 中的事件无需再加 on 字。\texttt{v-on} 有两种绑定事件的写法:

\begin{HTML}
<input type="button" value="绑定" v-on:click="方法">
<input type="button" value="绑定" @click="方法">
\end{HTML}

Vue 提供了一种简单的方式,用 \@ 替代 \texttt{v-on:}。

\texttt{v-on} 对应的值为 Vue 对象中的 \texttt{methods} 键所对应的函数名。 \texttt{methods} 也是一个键值对,其中键为函数名,值为函数定义\footnote{不要用箭头函数,因为没有 \texttt{this} 指针。}。

\lstinputlisting[language=HTML]{../../Examples/Basic/v-on.html}

\subsubsection{属性绑定: \texttt{v-show, v-if, v-bind}}

\noindent\textbf{\texttt{v-show}}

\texttt{v-show} 用于显示或隐藏某些元素,其对应的值被解析为 \texttt{true} 时才能显示。

\lstinputlisting[language=HTML]{../../Examples/Basic/v-show.html}

\noindent\textbf{\texttt{v-if}}

\texttt{v-if} 根据表达式的真假切换元素的显示和隐藏。操纵对象为 DOM 元素。注意,\texttt{v-show} 仅仅是操纵样式。\texttt{v-if} 操作 DOM 树需要更高的性能。频繁切换的元素使用 \texttt{v-show},要修改 DOM 树时才使用 \texttt{v-if}。

\lstinputlisting[language=HTML]{../../Examples/Basic/v-if.html}

\noindent\textbf{\texttt{v-bind}}

\texttt{v-bind} 用于设置元素的属性,将属性动态绑定到 Vue 对象的 \texttt{data} 中。和 \texttt{v-on} 一样,它有很多种写法:

\begin{HTML}
<img v-bind:src = "imgSrc">
<img v-bind:title = "imgTitle+'!!!'">
<img v-bind:class = "isActive?'active':''"> <!-- 三元判断 -->
<img v-bind:class = "{active:isActive}">    <!-- 三元判断的替代方法 -->
<img :src = "imgSrc">   <!-- 省略 v-bind -->
\end{HTML}

\lstinputlisting[language=HTML]{../../Examples/Basic/v-bind.html}

\subsubsection{表单元绑定: \texttt{v-for, v-on, v-bind}}
\noindent\textbf{\texttt{v-for}}

\texttt{v-for} 用于将数组元素载入列表。

\begin{HTML}
<li v-for="(item,index) in arr" >
    {{item}} {{index}}
</li>
\end{HTML}

\lstinputlisting[language=HTML]{../../Examples/Basic/v-for.html}

\noindent\textbf{\texttt{v-on} 补充}

\texttt{v-on} 对应的函数如果拥有参数,这样传值:

\begin{HTML}
<input type="button" @click="function(p1,p2)"/>
\end{HTML}

如果要将事件更精确地划分,可以使用点运算符:

\begin{HTML}
<input type="text" @keyup.enter="sayHi"/>
\end{HTML}

\noindent\textbf{\texttt{v-model}}

\texttt{v-model} 用于获取和设置表单元素的值(双向数据绑定)。这个过程是相互的,表单中的数据会被同步到 \texttt{data} 中,\texttt{data} 中的数据会被显示在表单中。

\lstinputlisting[language=HTML]{../../Examples/Basic/v-model.html}
\newpage